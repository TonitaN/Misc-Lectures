\documentclass[12pt]{article}
\usepackage{color}
\usepackage[newfloat]{minted}
\usepackage{amsfonts,latexsym,amsthm}
%\usepackage{times}
\usepackage[scaled=.92]{helvet}
\usepackage[all,2cell,knot,poly]{xy}
\usepackage{amssymb,array,amscd,enumerate}
\usepackage[T1,T2A]{fontenc}
\usepackage[utf8]{inputenc}
\usepackage[english,russian]{babel}
\usepackage{tikz,adjustbox}
%\usepackage{xparse}
\usetikzlibrary{automata, graphs,positioning, arrows,patterns,shapes.geometric,shapes.multipart,shapes.symbols,shapes.arrows}
\usetikzlibrary{arrows.meta,fit}

%\usepackage[cp1251]{inputenc}
%\usepackage{pscyr}
\def\logand{\mathrel{\&}}
\def\logor{\mathrel{\vee}}
\def\logimpl{\mathrel{\Rightarrow}}
\def\lognot{\mathop{\neg}}
\def\Kcm{\mathbf{K}}
\def\Icm{\mathbf{I}}
\def\Scm{\mathbf{S}}
\def\amodal{\mathop{\Box}}
\def\emodal{\mathop{\Diamond}}
\def\quantall#1{\mathop{\forall #1}}
\def\quantex#1{\mathop{\exists #1}} 
\def\either{\mathrm{either}\,}
\def\Leftt{\mathrm{Left}\;}
\def\Rightt{\mathrm{Right}\;}
\def\snd{\mathrm{snd}\;}
\def\fst{\mathrm{fst}\;}
\def\iff{\mathrel{\Leftrightarrow}}
%
\def\real{\mathop{\rm real}\nolimits}   
\def\Mult{\mathop{\rm Mult}\nolimits}   
\def\Inv{\mathop{\rm Inv}\nolimits}   
\def\round{\mathop{\rm round}\nolimits}
\newcommand{\skipper}{\hspace{-7pt}\textcolor{white}{\displaystyle !_{a_a}^{a^{a^{a^a}}}}}
\begin{document}

\section{Внутренние структуры данных}

Далее обозначаем алфавит констант греческой буквой $\Sigma$, алфавит переменных --- $\Xi$.

Базовые структуры данных алгоритма Ежа: переменная, константа и отрицательное условие.

\begin{itemize}
\item Поскольку множество переменных типа строка не меняется, то логично, что переменная представляется просто своим именем: например, $x$ представляется как \mintinline{refal}{(Var 'X')}.
\item Константы могут являться результатами сжатия блоков, поэтому логично представлять их парами: имя и индекс. Например, \mintinline{refal}{('D' 1)}.
\item Индексы дополнительно имеют расшифровки: мультимножества компонент их длины. Например, если \mintinline{refal}{('D' 1)} --- константа, хранящая блок $D^{i_1 + i_2+2}$, где $D$ --- константа исходного уравнения, то расшифровка будет: 

\mintinline{refal}{('D' 1) is (('D' 0) (i1 1)(i2 1)(const 2))}.
\item Отрицательное условие накладывается на переменную и имеет форму КНФ, где каждый литерал --- это отрицательное утверждение, чем не может кончаться или начинаться переменная, или утверждение о непустоте значения переменной. 

Например, \begin{minted}[breaklines,breakafter= ]{refal}
(OR 
    (not ('D' 1) ends (Var 'X')) 
    (not ('A' 2) starts (Var 'Y'))
)
\end{minted}
\end{itemize}

\section{Базовые операции интерактивного режима}
\begin{itemize}
\item Команда \mintinline{refal}{(PairComp C1 C2)}. Применить сжатие пар: 

\mintinline{refal}{<PairComp (/*Const1*/) (/*Const2*/) (/*EquationData*/)>}. 

Константы не должны совпадать. Результатом сжатия пар станет нумерованный набор новых уравнений с условиями (таким образом, сжатие пар преобразует одно уравнение к множеству уравнений).
\item Команда \mintinline{refal}{(BlockComp C)}. Применить сжатие блоков: 

\mintinline{refal}{<BlockComp (/*Const*/) (/*EquationData*/)>}. 

Аналогично, результат --- нумерованный набор новых уравнений с условиями.
\item Команда \mintinline{refal}{(Pick i)}. Выбрать из нумерованного списка уравнение, соответствующее номеру: \mintinline{refal}{<Pick /*Number*/ /*EquationSet*/>}. 

Действие позволяет перейти от множества уравнений к единственному.
\item Команда \mintinline{refal}{(Subst i1 (/* Multiset */))}. Осуществить подстановку мультимножества компонент, представленного в команде, вместо компоненты \mintinline{refal}{i1}.   
\end{itemize}

\section{Возможные результаты вычисления шага}

\begin{itemize}
\item Если в результате вычислений получилось уравнение вида $\varepsilon=\varepsilon$ или уравнение, которое сводится к нему после вычёркивания всех переменных, тогда объявить, что решение найдено.
\item Если в результате вычислений получилось уравнение, содержащее только переменные, причём хотя бы одна переменная имеет условие непустоты, тогда объявить, что на ветке нет минимального решения (остались только неявные сжатия).
\item Если команда противоречит условиям (например, требуется повторно сжимать блоки, с которых ничего начинаться не может, или требуется сжать блок, содержащий константу, не встречающуюся в уравнении) --- выдать сообщение о некорректном шаге.
\item В противном случае результатом вычислений станет одно или несколько состояний (т.е. уравнений в паре с условиями).
\end{itemize}

\section{Построение новых констант в \texttt{BlockComp} и \texttt{PairComp}}

Вопрос о том, ограничено ли множество используемых в процессе преобразования констант некоторой разумной (полиномиально зависящей от размера уравнения) величиной, пока открыт. Будем считать, что мы такое ограничение не ставим. Поэтому в дальнейшем нет смысла связывать новые имена констант со старыми, а лучше порождать их по очереди.

Чтобы код был написан полностью в функциональном стиле, достаточно хранить в общем состоянии последний используемый индекс блоков и последнюю использованную новую константу. Далее порождать на их основе очередную константу, например, пользуясь алфавитным сдвигом и комбинацией встроенных функций \mintinline{refal}{<Explode s.X>} -- \mintinline{refal}{<Implode e.X>} и \mintinline{refal}{<Numb e.N>} -- \mintinline{refal}{<Symb s.N>}.

\begin{itemize}
\item Функция \mintinline{refal}{<Explode s.X>} разбирает многобуквенную константу на символы и возвращает строку: \mintinline{refal}{Term} -> \mintinline{refal}{'Term'}.
\item Функция \mintinline{refal}{<Implode e.X>} собирает символы в многобуквенную константу (не любая последовательность символов может ей стать, выбирается самая длинная подпоследовательность значения \mintinline{refal}{e.X}): \mintinline{refal}{'Term 123'} -> \mintinline{refal}{Term '123'}.
\item Функция \mintinline{refal}{<Symb s.X>} разбирает число на символы и возвращает строку: \mintinline{refal}{123} -> \mintinline{refal}{'123'}. Функция \mintinline{refal}{<Numb e.X>} действует обратным образом.
\end{itemize}

Простейший код для порождения новых индексов блоков всего лишь увеличивает индекс константы. Константой здесь считаем литерал (многобуквенный терм без кавычек), который начинается на букву, а следом за ней обязательно идёт цифра. Например, \mintinline{refal}{i19} (не путаем со строкой \mintinline{refal}{'i19'}).

\begin{minted}{refal}
GetNewIndex {
 s.OldIndex
 , <Explode s.OldIndex> : s.Letter e.Digits
 , <Symb <Add <Numb e.Digits> 1>> : e.NewNumber
  = <Implode s.Letter e.NewNumber>;
}
\end{minted}

Улучшением этого базового варианта может быть вариант, который меняет сначала букву (выбирая в строке возможных вариантов ближайшую справа от текущей), а если не задействованных букв уже не осталось --- то цифру, возвращая указатель на букву в начало алфавита.

\begin{minted}[breaklines, linenos]{refal}
/* В этом варианте есть тонкость: начальным буквам стоит присвоить индексы с нулём, а порождённым - начиная с единицы, чтобы не путаться. Реализацию этой тонкости оставим за кадром ТЗ. */
GetNewIndexUpdated {
 s.OldIndex
 , <Explode s.OldIndex> : s.Letter e.Digits
 , <CharAplhabet> : {
    e.A1 s.Letter s.NextLetter e.A2 /* Можно выбрать следующую по очереди букву */
        = <Implode s.NextLetter e.Digits>;
    s.FirstLetter e.A1 s.Letter /* Буква уже последняя --- переходим на первую и увеличиваем число */
    , <Symb <Add <Numb e.Digits> 1>> : e.NewNumber
        = <Implode s.FirstLetter e.NewNumber>;
  };
}
\end{minted}

Заметим, что эти же приёмы можно использовать для порождения не только индексов блоков, но и букв --- то есть пар вида \mintinline{refal}{(s.Name s.Number)}. Всё, что нужно будет поменять --- это не делать первый вызов функции \mintinline{refal}{<Explode s.OldIndex>} с последующей сборкой символов в число (и обратное ему преобразование), а просто сдвигать букву \mintinline{refal}{s.Name} и увеличивать числовой индекс \mintinline{refal}{s.Number}.
 
\section{Нулевые значения переменных}
 
В дальнейшем подразумевается следующее допущение. Все извлечённые блоки по умолчанию полагаются непустыми --- пустыми их может сделать только явная подстановка \mintinline{refal}{(SubstIndex ik ((Const 0)))} (или композиция таких подстановок), за которую отвечает пользователь. Более того, все извлечённые блоки по умолчанию предполагаются неравными --- в противном случае от пользователя ожидается подстановка, которая явно утверждает обратное.

Однако в случае вызова \mintinline{refal}{<PairComp /* args */>} может получиться, что потребуется автоматически разобрать и случай пустой подстановки в переменную, чтобы выявить все возможности сжатия перекрёстных или явных вхождений. Например, как в следующем уравнении (назовём его Уравнением 1): 

\begin{minted}{refal}
(AreEqual 
   ((Var 'X') ('B' 0) (Var 'Y') ('A' 0))
   ((Var 'Y') (Var 'Y') ('B' 0) ('A' 0) (Var 'X'))
)
(
   (OR (not empty (Var 'X')))
)
(/* Блок уравнений на компоненты пуст */)
\end{minted}
 
Вызов \mintinline{refal}{<PairComp ('B' 0) ('A' 0) (/* Уравнение 1 */)>} порождает только разбор случаев по переменной \verb|Y|. Полагаем, что новой константой, сжимающей \mintinline{refal}{('B' 0)('A' 0)}, является ('A' 1). 

\begin{enumerate}
\item Случай, когда \verb|Y| кончается на \mintinline{refal}{('B' 0)} и начинается на \mintinline{refal}{('A' 0)}:

\begin{minted}{refal}
(AreEqual 
   ((Var 'X') ('A' 1) (Var 'Y') ('A' 1))
   (('A' 0) (Var 'Y') ('A' 1) (Var 'Y') ('B' 0)('A' 1) (Var 'X'))
)
(
   (OR (not empty (Var 'X')))
)
((('A' 1) is (('A' 0) (const 1))(('B' 0) (const 1))))
\end{minted}
 
\item Случай, когда \verb|Y| кончается на \mintinline{refal}{('B' 0)}, но не начинается на \mintinline{refal}{('A' 0)}:

\begin{minted}{refal}
(AreEqual 
   ((Var 'X') ('B' 0) (Var 'Y') ('A' 1))
   ((Var 'Y') ('B' 0) (Var 'Y') ('B' 0)('A' 1) (Var 'X'))
)
(
   (OR (not empty (Var 'X')))
   (OR (not ('A' 0) starts (Var 'Y')))
)
((('A' 1) is (('A' 0) (const 1))(('B' 0) (const 1))))
\end{minted}

\item Случай, когда \verb|Y| не кончается на \mintinline{refal}{('B' 0)}, но начинается на \mintinline{refal}{('A' 0)}:

\begin{minted}{refal}
(AreEqual 
   ((Var 'X') ('A' 1) (Var 'Y') ('A' 0))
   (('A' 0) (Var 'Y') ('A' 0) (Var 'Y') ('A' 1) (Var 'X'))
)
(
   (OR (not empty (Var 'X')))
   (OR (not ('B' 0) ends (Var 'Y')))
)
((('A' 1) is (('A' 0) (const 1))(('B' 0) (const 1))))
\end{minted}

\item Случай, когда \verb|Y| не кончается на \mintinline{refal}{('B' 0)} и не начинается на \mintinline{refal}{('A' 0)}:

\begin{minted}{refal}
(AreEqual 
   ((Var 'X') ('B' 0) (Var 'Y') ('A' 0))
   ((Var 'Y') (Var 'Y') ('A' 1) (Var 'X'))
)
(
   (OR (not empty (Var 'X')))
   (OR (not ('A' 0) starts (Var 'Y')))
   (OR (not ('B' 0) ends (Var 'Y')))
)
((('A' 1) is (('A' 0) (const 1))(('B' 0) (const 1))))
\end{minted}

\item Случай, когда \verb|Y| пуст:

\begin{minted}{refal}
(AreEqual 
   ((Var 'X') ('A' 1))
   (('A' 1) (Var 'X'))
)
(
   (OR (not empty (Var 'X')))
)
(/* Блок уравнений на компоненты пуст */)
\end{minted}
\end{enumerate}

\begin{figure}[!htb]\hspace*{-4em}
\adjustbox{max width=0.8\paperwidth}{
\begin{tikzpicture}[
level 0/.style = {sibling distance = 11em,level distance = 5em},
level 0/.style = {sibling distance = 11em,level distance = 4em},
level 1/.style = {sibling distance = 6em,level distance = 5em},
level 2/.style = {sibling distance = 13em,level distance = 8em},
]


\tikzstyle{equation}=[rectangle, draw opacity=0.8,rounded corners,thick,draw=black!75,top color=blue!10, bottom color=white,minimum size=5mm, font =\footnotesize]
\tikzstyle{eqset}=[draw,cloud,dotted,opacity=0.9]
\tikzstyle{failequation}=[rectangle split,rectangle split parts=2,
    rounded corners,
    thick,
    draw=black!50,
    draw opacity=0.8,
    minimum size=5mm, font =\footnotesize
    ,append after command={\pgfextra
        \fill[left color=red!35!gray, right color=red!55!gray]
    (\tikzlastnode.one west) 
    [rounded corners] |- (\tikzlastnode.north) -| (\tikzlastnode.one east) 
    [sharp corners]   |- (\tikzlastnode.one split) -| cycle;
        \fill[top color=gray!15, bottom color=white]
    (\tikzlastnode.two west) 
    [rounded corners] |- (\tikzlastnode.south) -| (\tikzlastnode.two east)  
    [sharp corners]   |- (\tikzlastnode.one split) -| cycle;
                                    \endpgfextra}]
\tikzstyle{fullequation}=[rectangle split,rectangle split parts=2,
    rounded corners,
    thick,
    draw=black!50,
    draw opacity=0.8,
    minimum size=5mm, font =\footnotesize
    ,append after command={\pgfextra
        \fill[left color=blue!35, right color=blue!25]
    (\tikzlastnode.one west) 
    [rounded corners] |- (\tikzlastnode.north) -| (\tikzlastnode.one east) 
    [sharp corners]   |- (\tikzlastnode.one split) -| cycle;
        \fill[top color=blue!10, bottom color=white]
    (\tikzlastnode.two west) 
    [rounded corners] |- (\tikzlastnode.south) -| (\tikzlastnode.two east)  
    [sharp corners]   |- (\tikzlastnode.one split) -| cycle;
                                    \endpgfextra}]
\tikzstyle{chosenequation}=[rectangle split,rectangle split parts=2,
    rounded corners,
    thick,
    double, double distance=1.5pt,
    draw=black!50,
    draw opacity=0.8,
    minimum size=5mm, font =\footnotesize
    ,append after command={\pgfextra
        \fill[left color=yellow!50, right color=yellow!35]
    (\tikzlastnode.one west) 
    [rounded corners] |- (\tikzlastnode.north) -| (\tikzlastnode.one east) 
    [sharp corners]   |- (\tikzlastnode.one split) -| cycle;
        \fill[top color=yellow!20, bottom color=white]
    (\tikzlastnode.two west) 
    [rounded corners] |- (\tikzlastnode.south) -| (\tikzlastnode.two east)  
    [sharp corners]   |- (\tikzlastnode.one split) -| cycle;
                                    \endpgfextra}]
\tikzstyle{action}=[single arrow,shape border rotate=270,inner sep=0.1ex,thick,draw=black!75,top color=red!10, bottom color=white,minimum height=4em,single arrow head extend=1em,single arrow head indent=0.5ex, font =\footnotesize,,opacity=1]
\tikzstyle{subst}=[shape aspect=4,diamond, thick,draw=black!75, top color=green!5, bottom color=white,minimum size=5mm] 

\node (0B1) [chosenequation]{\nodepart{one}$\mathbf{X}B_{0}\mathbf{Y}A_{0} = \mathbf{Y}\mathbf{Y}B_{0}A_{0}\mathbf{X}$\nodepart{two}$\begin{array}{l}\mathbf{X}\neq\varepsilon\\\mathrm{No\;conditions}\\\end{array}$}
child {node [action]{$\begin{array}{c}\mathtt{PairComp}\\B_{0}\;A_{0}\end{array}$} edge from parent [draw opacity = 0]
child {node (2B1) [fullequation]{\nodepart{one}$\mathbf{X}A_{1}\mathbf{Y}A_{1} = A_{0}\mathbf{Y}A_{1}\mathbf{Y}B_{0}A_{1}\mathbf{X}$\nodepart{two}$\begin{array}{l}\mathbf{X}\neq\varepsilon\\A_{1}:=A_{0}B_{0}\end{array}$}edge from parent [draw opacity = 0]
}
child {node (2B2) [fullequation]{\nodepart{one}$\mathbf{X}B_{0}\mathbf{Y}A_{1} = \mathbf{Y}B_{0}\mathbf{Y}B_{0}A_{1}\mathbf{X}$\nodepart{two}$\begin{array}{l}\mathbf{X}\neq\varepsilon\\\mathbf{Y}\neq{}A_{0}\mathbf{Y_{S}}\\A_{1}:=A_{0}B_{0}\end{array}$}edge from parent [draw opacity = 0]
}
child {node (2B3) [fullequation]{\nodepart{one}$\mathbf{X}A_{1}\mathbf{Y}A_{0} = A_{0}\mathbf{Y}A_{0}\mathbf{Y}A_{1}\mathbf{X}$\nodepart{two}$\begin{array}{l}\mathbf{X}\neq\varepsilon\\\mathbf{Y}\neq{}\mathbf{Y_{P}}B_{0}\\A_{1}:=A_{0}B_{0}\end{array}$}edge from parent [draw opacity = 0]
}
child {node (2B4) [fullequation]{\nodepart{one}$\mathbf{X}B_{0}\mathbf{Y}A_{0} = \mathbf{Y}\mathbf{Y}A_{1}\mathbf{X}$\nodepart{two}$\begin{array}{l}\mathbf{X}\neq\varepsilon\\\textcolor{red}{\mathbf{Y}\neq\varepsilon}\\\mathbf{Y}\neq{}A_{0}\mathbf{Y_{S}}\\\mathbf{Y}\neq{}\mathbf{Y_{P}}B_{0}\\A_{1}:=A_{0}B_{0}\end{array}$}edge from parent [draw opacity = 0]
}
child {node (2B5) [fullequation]{\nodepart{one}$\mathbf{X}A_{1} = A_{1}\mathbf{X}$\nodepart{two}$\begin{array}{l}\mathbf{X}\neq\varepsilon\\\mathrm{No\;conditions}\\\end{array}$}edge from parent [draw opacity = 0]
}
child {node [eqset,inner sep=-3em,cloud puffs=70,aspect=6,fit=(2B1)(2B2)(2B3)(2B4)(2B5)] {}
};};
\end{tikzpicture}
}
\caption{Пустая подстановка при сжатии пар}
\label{fig::test1}
\end{figure}

Все подстановки показаны на Рисунке~\ref{fig::test1}.

Последний случай принципиально отличается от всех предыдущих и не может быть сведён к ним. Кажется, что он похож на случай 4, но в случае 4 есть очевидное противоречие, если рассмотреть его образ в линейной целочисленной арифметике по модулю $A_1$, поскольку предполагается, что буквы $B_0$ и $A_0$, явно входящие в уравнение, уже точно не принадлежат никакому вхождению $B_0 A_0$. В случае же 5 противоречий нет, и он является единственным, приводящим к решению (после подстановки сжатия блоков $A_1$). 

\section{Извлечение пар в PairComp}

Перекрёстные и явные пары в \verb|PairComp|($\gamma_1$,$\gamma_2$) могут появляться вследствие следующих подстановок.
\begin{itemize}
\item элементарная подстановка $X\mapsto X\gamma_1$, $X\mapsto \gamma_2 X$;
\item композиция двух элементарных подстановок (например, при извлечении $\gamma_1\gamma_2$) из границы между $X$ и $Y$;
\item пустая подстановка $X\mapsto \varepsilon$ (см. выше).
\end{itemize}
 
Скажем, что элементарная подстановка существенная, если в результате её применения (без иных действий) появляется новая явная пара $\gamma_1\gamma_2$. 
 
Скажем, что композиция $\tau_i$ существенная, если в результате её применения появляется явная пара, которая не появляется при применении элементов $\tau_i$ по отдельности.

Пусть $\tau = \sigma_2 \circ \sigma_1$ существенна (все подстановки независимые, порядок может быть любым, существенны лишь сами $\sigma_i$). Возможны следующие случаи:

\begin{itemize}
\item $\sigma_1$ и $\sigma_2$ тоже существенны. Тогда относительно $\sigma_i$ придётся перебрать все четыре варианта комбинаций.
\item $\sigma_1$ существенна, $\sigma_2$ нет. Тогда вариант применения только $\sigma_2$ и вариант применения ни одной из подстановок эквиваленты (и описываются отрицательной рестрикцией на $\sigma_1$), и нет смысла выделять оба. Остаются только три варианта из четырёх.
\item ни одна из компонент не существенна. Тогда случаи их применения по отдельности оба эквивалентны отсутствию подстановок (и соответствуют отрицательной рестрикции, представляющей собой дизъюнкцию), и остаётся только два варианта.  
\end{itemize}

После построения базовых вариантов по каждой из композиций, а также базовых вариантов по существенным элементарным подстановкам, не вошедшим ни в одну композицию, строятся прямые произведения. При этом некоторые из произведений будут отброшены как повторные, а некоторые --- как противоречивые; кроме того, могут происходить нетривиальные преобразования навешиваемых отрицательных условий (упрощение 2КНФ). Всю логику 2-КНФ здесь реализовывать нет необходимости, потому что все элементы дизъюнкций, состояющих больше, чем из одного дизъюнкта, отрицательные. Поэтому достаточно <<протолкнуть>> (т.е. подставить) все положительные значения подстановок в дизъюнкты (для упрощения и нахождения противоречий); и выбросить 2-дизъюнкты, которые поглощаются 1-дизъюнктами.

Рассмотрим, например, уравнение \verb|bXYa = XbZY|, к которому применяется \verb|PairComp(ab)|. Выделяется существенная подстановка $\sigma_1$, \verb|X|$\sigma_1=$\verb|Xa|, а также элементы композиций $\sigma_2$ и $\sigma_3$: \verb|Y|$\sigma_2=$\verb|bY|, \verb|Z|$\sigma_3=$\verb|Za|. По композиции $\sigma_2\circ\sigma_1$ получается три случая.

\begin{itemize}
\item (1-1) \verb|X|$=$\verb|Xa|$\logand$\verb|Y|$=$\verb|bY|;
\item (1-2) \verb|X|$=$\verb|Xa|$\logand$\verb|Y|$\neq$\verb|bY|;
\item (1-3) \verb|X|$\neq$\verb|Xa|
\end{itemize}  
 
По композиции $\sigma_2\circ\sigma_3$ случая всего два.

\begin{itemize}
\item (2-1) \verb|Z|$=$\verb|Za|$\logand$\verb|Y|$=$\verb|bY|;
\item (2-2) \verb|Z|$\neq$\verb|Xa|$\logor$\verb|Y|$\neq$\verb|bY|.
\end{itemize}  
 
Рассмотрим все их произведения. 

\begin{itemize}
\item (1-1)$\times$(2-1) \verb|X|$=$\verb|Xa|$\logand$\verb|Y|$=$\verb|bY|$\logand$\verb|Z|$=$\verb|Za|;
\item (1-1)$\times$(2-2) \verb|X|$=$\verb|Xa|$\logand$\verb|Y|$=$\verb|bY|$\logand$\verb|Z|$\neq$\verb|Xa| (здесь дизъюнкция вынуждает выполниться первый дизъюнкт, т.к. второй вступает в противоречие с положительным условием);
\item (1-2)$\times$(2-1) противоречие;
\item (1-2)$\times$(2-2) \verb|X|$=$\verb|Xa|$\logand$\verb|Y|$\neq$\verb|bY|. А здесь дизъюнкция тривиально выполнена, поэтому добавлять её не нужно.
\item (1-3)$\times$(2-1) \verb|X|$\neq$\verb|Xa|$\logand$\verb|Z|$=$\verb|Za|$\logand$\verb|Y|$=$\verb|bY|;
\item (1-3)$\times$(2-2) \verb|X|$\neq$\verb|Xa|$\logand$(\verb|Z|$\neq$\verb|Xa|$\logor$\verb|Y|$\neq$\verb|bY|).
\end{itemize}  

Осталось пять вариантов (Рисунок~\ref{fig::test2}). 

\begin{figure}[!htb]
\hspace*{-4em}\adjustbox{max width=0.8\paperwidth}{
\begin{tikzpicture}[
level 0/.style = {sibling distance = 10em,level distance = 4em},
level 0/.style = {sibling distance = 10em,level distance = 3em},
level 1/.style = {sibling distance = 6em,level distance = 5em},
level 2/.style = {sibling distance = 12em,level distance = 7em},
]


\tikzstyle{equation}=[rectangle, draw opacity=0.8,rounded corners,thick,draw=black!75,top color=blue!10, bottom color=white,minimum size=5mm, font =\footnotesize]
\tikzstyle{eqset}=[draw,cloud,dotted,opacity=0.9]
\tikzstyle{failequation}=[rectangle split,rectangle split parts=2,
    rounded corners,
    thick,
    draw=black!50,
    draw opacity=0.8,
    minimum size=5mm, font =\footnotesize
    ,append after command={\pgfextra
        \fill[left color=red!35!gray, right color=red!55!gray]
    (\tikzlastnode.one west) 
    [rounded corners] |- (\tikzlastnode.north) -| (\tikzlastnode.one east) 
    [sharp corners]   |- (\tikzlastnode.one split) -| cycle;
        \fill[top color=gray!15, bottom color=white]
    (\tikzlastnode.two west) 
    [rounded corners] |- (\tikzlastnode.south) -| (\tikzlastnode.two east)  
    [sharp corners]   |- (\tikzlastnode.one split) -| cycle;
                                    \endpgfextra}]
\tikzstyle{fullequation}=[rectangle split,rectangle split parts=2,
    rounded corners,
    thick,
    draw=black!50,
    draw opacity=0.8,
    minimum size=5mm, font =\footnotesize
    ,append after command={\pgfextra
        \fill[left color=blue!35, right color=blue!25]
    (\tikzlastnode.one west) 
    [rounded corners] |- (\tikzlastnode.north) -| (\tikzlastnode.one east) 
    [sharp corners]   |- (\tikzlastnode.one split) -| cycle;
        \fill[top color=blue!10, bottom color=white]
    (\tikzlastnode.two west) 
    [rounded corners] |- (\tikzlastnode.south) -| (\tikzlastnode.two east)  
    [sharp corners]   |- (\tikzlastnode.one split) -| cycle;
                                    \endpgfextra}]
\tikzstyle{chosenequation}=[rectangle split,rectangle split parts=2,
    rounded corners,
    thick,
    double, double distance=1.5pt,
    draw=black!50,
    draw opacity=0.8,
    minimum size=5mm, font =\footnotesize
    ,append after command={\pgfextra
        \fill[left color=yellow!50, right color=yellow!35]
    (\tikzlastnode.one west) 
    [rounded corners] |- (\tikzlastnode.north) -| (\tikzlastnode.one east) 
    [sharp corners]   |- (\tikzlastnode.one split) -| cycle;
        \fill[top color=yellow!20, bottom color=white]
    (\tikzlastnode.two west) 
    [rounded corners] |- (\tikzlastnode.south) -| (\tikzlastnode.two east)  
    [sharp corners]   |- (\tikzlastnode.one split) -| cycle;
                                    \endpgfextra}]
\tikzstyle{action}=[single arrow,shape border rotate=270,inner sep=0.1ex,thick,draw=black!75,top color=red!10, bottom color=white,minimum height=4em,single arrow head extend=1em,single arrow head indent=0.5ex, font =\footnotesize,,opacity=1]
\tikzstyle{subst}=[shape aspect=4,diamond, thick,draw=black!75, top color=green!5, bottom color=white,minimum size=5mm] 

\node (0B1) [chosenequation]{\nodepart{one}$b_{0}\mathbf{X}\mathbf{Y}a_{0} = \mathbf{X}b_{0}\mathbf{Z}\mathbf{Y}$\nodepart{two}$\begin{array}{l}\mathrm{No\;constraints}\\\mathrm{No\;conditions}\\\end{array}$}
child {node [action]{$\begin{array}{c}\mathtt{PairComp}\\a_{0}\;b_{0}\end{array}$} edge from parent [draw opacity = 0]
child {node (2B1) [fullequation]{\nodepart{one}$b_{0}\mathbf{X}c_{0}\mathbf{Y}a_{0} = \mathbf{X}c_{0}\mathbf{Z}c_{0}\mathbf{Y}$\nodepart{two}$\begin{array}{l}\mathrm{No\;constraints}\\c_{0}:=a_{0}b_{0}\end{array}$}edge from parent [draw opacity = 0]
}
child {node (2B2) [fullequation]{\nodepart{one}$b_{0}\mathbf{X}c_{0}\mathbf{Y}a_{0} = \mathbf{X}c_{0}\mathbf{Z}b_{0}\mathbf{Y}$\nodepart{two}$\begin{array}{l}\mathbf{Z}\neq{}\mathbf{Z_{P}}a_{0}\\c_{0}:=a_{0}b_{0}\end{array}$}edge from parent [draw opacity = 0]
}
child {node (2B3) [fullequation]{\nodepart{one}$b_{0}\mathbf{X}a_{0}\mathbf{Y}a_{0} = \mathbf{X}c_{0}\mathbf{Z}\mathbf{Y}$\nodepart{two}$\begin{array}{l}\mathbf{Y}\neq{}b_{0}\mathbf{Y_{S}}\\c_{0}:=a_{0}b_{0}\end{array}$}edge from parent [draw opacity = 0]
}
child {node (2B4) [fullequation]{\nodepart{one}$b_{0}\mathbf{X}b_{0}\mathbf{Y}a_{0} = \mathbf{X}b_{0}\mathbf{Z}c_{0}\mathbf{Y}$\nodepart{two}$\begin{array}{l}\mathbf{X}\neq{}\mathbf{X_{P}}a_{0}\\c_{0}:=a_{0}b_{0}\end{array}$}edge from parent [draw opacity = 0]
}
child {node (2B5) [fullequation]{\nodepart{one}$b_{0}\mathbf{X}\mathbf{Y}a_{0} = \mathbf{X}b_{0}\mathbf{Z}\mathbf{Y}$\nodepart{two}$\begin{array}{l}\mathbf{X}\neq{}\mathbf{X_{P}}a_{0}\\\mathbf{Y}\neq{}b_{0}\mathbf{Y_{S}}\vee\mathbf{Z}\neq{}\mathbf{Z_{P}}a_{0}\\\mathrm{No\;conditions}\\\end{array}$}edge from parent [draw opacity = 0]
}
child {node [eqset,inner sep=-3em,cloud puffs=70,aspect=6,fit=(2B1)(2B2)(2B3)(2B4)(2B5)] {}
};};
\end{tikzpicture}
}
\caption{Порождение случаев при сжатии пар}
\label{fig::test2}
\end{figure}
 
 \medskip
\subsection{Существенные пустые подстановки}

Скажем, что подстановка $\eta:$\verb|Y|$\mapsto\varepsilon$ существенная, если в результате её применения становится возможно построить новую явную или перекрёстную пару.

Скажем, что композиция пустых подстановок $\theta$ существенная, если в результате её применения становится возможно построить новую явную или перекрёстную пару, которая не строится при применении любого подмножества подстановок, участвующих в композиции.

Определим критерии существенности пустых подстановок и композиции пустых подстановок при сжатии пары $\gamma_1 \gamma_2$.
\begin{itemize}
\item Пусть часть уравнения имеет вид $\Phi_1\gamma_1\Psi\gamma_2\Phi_2$, где $\Psi$ непуста и принадлежит алфавиту переменных $\Xi$. Тогда композиция подстановок в $\Psi$, обращающая $\Psi$ в $\varepsilon$, считается существенной. Пример: в уравнении $\mathtt{YbZ\textcolor{red}{a}YZ\textcolor{red}{b} = XaYWYZ}$ при сжатии \verb|ab| композиция $\theta_1 = \eta_1\circ\eta_2$, где $\eta_1: \mathtt{Y}\mapsto\varepsilon$, $\eta_2: \mathtt{Z}\mapsto\varepsilon$, существенна, поскольку порождает новую явную пару, выделенную красным.
\item Пусть часть уравнения имеет вид $\Phi_1\gamma_1\Psi X\Phi_2$, где $X$ --- переменная, а $\Psi$ непуста и принадлежит алфавиту $\Xi\setminus\{X\}$. Тогда композиция подстановок в $\Psi$, обращающая $\Psi$ в $\varepsilon$, считается существенной. Пример: в уравнении $\mathtt{YbZ{a}YZ{b} = X\textcolor{red}{a}YWY\textcolor{red}{Z}}$ при сжатии \verb|ab| композиция $\theta_2 = \eta_1\circ\eta_3$, где $\eta_1: \mathtt{Y}\mapsto\varepsilon$, $\eta_3: \mathtt{W}\mapsto\varepsilon$, существенна, поскольку порождает новую перекрёстную пару, если положить $\mathtt{Z}\mapsto\mathtt{bZ}$. Аналогичным свойством обладает и подстановка $\eta_1$ сама по себе: она порождает уравнение \verb|bZaZb = XaWZ|, где появляется новое соседство буквы \verb|a| с переменными \verb|Z| и \verb|Y| справа от неё.
\item Критерий, симметричный предыдущему. Пусть часть уравнения имеет вид $\Phi_1 X\Psi \gamma_2\Phi_2$, где $X$ --- переменная, а $\Psi$ непуста и принадлежит алфавиту $\Xi\setminus\{X\}$. Тогда композиция подстановок в $\Psi$, обращающая $\Psi$ в $\varepsilon$, считается существенной. В том же уравнении $\mathtt{YbZ{a}\textcolor{red}{Y}Z\textcolor{red}{b} = X{a}YWY{Z}}$ при сжатии \verb|ab| подстановка $\eta_2$ существенна в этом смысле.
\item Последний случай: новое соседство между переменными. Пусть часть уравнения имеет вид $\Phi_1 X_1\Psi X_2\Phi_2$, где $X_i$ --- переменные (возможно, равные), а $\Psi$ непуста и принадлежит алфавиту $\Xi\setminus\{X_1,X_2\}$. Тогда композиция подстановок в $\Psi$, обращающая $\Psi$ в $\varepsilon$, считается существенной. В уравнении $\mathtt{YbZ{a}{Y}Z{b} = X{a}\textcolor{blue}{Y}\textcolor{red}{W}\textcolor{blue}{Y}\textcolor{red}{Z}}$ при сжатии \verb|ab| существенными по этому критерию будут подстановки $\eta_1$ (порождает смежную пару, выделенную красным) и $\eta_3$ (порождает смежную пару, выделенную синим).

\end{itemize}

Кажется, что вышеперечисленные случаи порождают огромное число композиций, в которых, в том числе, будут отрицательные условия дизъюнктами не в 2-КНФ. Однако можно заметить\footnote{И потом доказать --- это TODO остаётся за мной, но если вы хотите, то можете попробовать.}, что, в отличие от случая непустых подстановок, если композиция $\theta$ является существенной, то все подстановки в ней являются существенными. Поэтому имеет смысл максимизировать фрагменты $\Psi$, обладающие требуемыми свойствами, чтобы выделить из них все необходимые для разбора случаев пустые подстановки, и дальше --- комбинировать их между собой всеми способами и применять к результатам соответствующих подстановок уже только перебор по непустым подстановкам. Это приведёт к определённого рода избыточности: например, комбинация $\eta_1\circ\eta_2\circ\eta_3$ порождает уравнение \verb|bab = Xa|, которое является частным случаем уравнения \verb|bab = XaW|  при условии, что \verb|W| не начинается с \verb|b|. Кажется, такая избыточность излечима, если пользоваться леммой и запускать перебор существенно пустых подстановок по одной, рекурсивно\footnote{При этом на все переменные, которые не обращаются в пустое слово на данном рекурсивном шаге, нужно объявлять непустыми в дополнительной рестрикции. Такая рестрикция выделена красным на Рисунке~\ref{fig::test1}.}.
 
\section{Пример}

Покажем пример работы цепочки операций на следующем примере.

Исходное уравнение: 

\begin{minted}[breaklines,breakafter= ]{refal}
(AreEqual 
 ((Var 'X') ('A' 0) ('A' 0)) 
 (('B' 0) (Var 'Y'))
)
(/* Блок условий пуст */)
(/* Блок уравнений на компоненты пуст */)
\end{minted}

\begin{itemize}

\item Применяем \mintinline{refal}{(BlockComp ('A' 0))}. Получаем набор уравнений:
\begin{enumerate}
\item Случай, когда все переменные коллапсируют в блоки:
\begin{minted}[breaklines,breakafter= ]{refal}
(AreEqual 
 (('A' 1)) 
 (('B' 0) ('A' 2))
)
(/* Блок условий пуст */)
( 
 (('A' 1) is ('A' (i1 1) (const 2))) 
 (('A' 2) is ('A' (i2 1) (const 0))) 
)
\end{minted}
\item В блок коллапсирует только \verb|X|:
\begin{minted}[breaklines,breakafter= ]{refal}
(AreEqual 
 (('A' 1)) 
 (('B' 0) ('A' 2) (Var 'Y') ('A' 3))
)
(
  (OR (not empty (Var 'Y')))
  (OR (not ('A' 1) ends (Var 'Y'))
  (OR (not ('A' 2) ends (Var 'Y'))
  (OR (not ('A' 3) ends (Var 'Y'))
  (OR (not ('A' 1) starts (Var 'Y'))
  (OR (not ('A' 2) starts (Var 'Y'))
  (OR (not ('A' 3) starts (Var 'Y'))
)
( 
 (('A' 1) is (('A' 0) (i1 1) (const 2))) 
 (('A' 2) is (('A' 0) (i2 1) (const 0))) 
 (('A' 3) is (('A' 0) (i3 1) (const 0))) 
)
\end{minted}
\item В блок коллапсирует только \verb|Y|:
\begin{minted}[breaklines,breakafter= ]{refal}
(AreEqual 
 (('A' 1) (Var 'X') ('A' 2)) 
 (('B' 0) ('A' 3))
)
(
  (OR (not empty (Var 'X')))
  (OR (not ('A' 1) ends (Var 'X'))
  (OR (not ('A' 2) ends (Var 'X'))
  (OR (not ('A' 3) ends (Var 'X'))
  (OR (not ('A' 1) starts (Var 'X'))
  (OR (not ('A' 2) starts (Var 'X'))
  (OR (not ('A' 3) starts (Var 'X'))
)
( 
 (('A' 1) is (('A' 0) (i1 1) (const 0))) 
 (('A' 2) is (('A' 0) (i2 1) (const 2))) 
 (('A' 3) is (('A' 0) (i3 1) (const 0))) 
)
\end{minted}
\item Ничего не коллапсирует:
\begin{minted}[breaklines,breakafter= ]{refal}
(AreEqual 
 (('A' 1) (Var 'X') ('A' 2)) 
 (('B' 0) ('A' 3) (Var 'Y') ('A' 4))
)
(
  (OR (not empty (Var 'X')))
  (OR (not empty (Var 'Y')))
  (OR (not ('A' 1) ends (Var 'X'))
  (OR (not ('A' 2) ends (Var 'X'))
  (OR (not ('A' 3) ends (Var 'X'))
  (OR (not ('A' 4) ends (Var 'X'))
  (OR (not ('A' 1) starts (Var 'X'))
  (OR (not ('A' 2) starts (Var 'X'))
  (OR (not ('A' 3) starts (Var 'X'))
  (OR (not ('A' 4) starts (Var 'X'))
  (OR (not ('A' 1) ends (Var 'Y'))
  (OR (not ('A' 2) ends (Var 'Y'))
  (OR (not ('A' 3) ends (Var 'Y'))
  (OR (not ('A' 4) ends (Var 'Y'))
  (OR (not ('A' 1) starts (Var 'Y'))
  (OR (not ('A' 2) starts (Var 'Y'))
  (OR (not ('A' 3) starts (Var 'Y'))
  (OR (not ('A' 4) starts (Var 'Y'))
)
 (('A' 1) is (('A' 0) (i1 1) (const 0))) 
 (('A' 2) is (('A' 0) (i2 1) (const 2))) 
 (('A' 3) is (('A' 0) (i3 1) (const 0))) 
 (('A' 4) is (('A' 0) (i4 1) (const 0))) 
)
\end{minted}

\end{enumerate}
\item Выбираем четвёртый случай: \mintinline{refal}{(Pick 4)}. Осуществляем ряд подстановок индексов: 

\mintinline{refal}{(Subst i1 ((const 0)))}

\mintinline{refal}{(Subst i3 ((const 0)))}

\mintinline{refal}{(Subst i4 ((i2 1) (const 2)))}

Интерактивный режим сам сократит одинаковые суффиксы и удалит ненужные зависимости, ведь констант \mintinline{refal}{('A' k)} после этого в уравнении не останется:

\begin{minted}[breaklines,breakafter= ]{refal}
(AreEqual 
 ((Var 'X')) 
 (('B' 0) (Var 'Y'))
)
(/* Тут ничего нет, кроме условий на непустоту: 
    все упоминания констант, связанных с 'A', 
    исчезли из уравнения, и нет никаких связей этих 
    констант ни с чем, кроме как с собой, в блоке
    индексов, значит, и условия на них уже не нужны */
  (OR (not empty (Var 'X')))
  (OR (not empty (Var 'Y')))    
 )
(/* Тут тоже ничего нет --- всё сократилось */)
\end{minted}
\item Сжимаем блоки \mintinline{refal}{('B' 0)}. Опять получаем четыре случая.
\begin{enumerate}
\item Случай, когда все переменные коллапсируют в блоки:
\begin{minted}[breaklines,breakafter= ]{refal}
(AreEqual 
 (('B' 1)) 
 (('B' 2))
)
(/* Блок условий пуст */)
( 
 (('B' 1) is ('B' (i1 1) (const 0))) 
 (('B' 2) is ('B' (i2 1) (const 1))) 
)
\end{minted}
\item В блок коллапсирует только \verb|X|:
\begin{minted}[breaklines,breakafter= ]{refal}
(AreEqual 
 (('B' 1)) 
 (('B' 2) (Var 'Y') ('B' 3))
)
(
  (OR (not empty (Var 'Y')))
  (OR (not ('B' 1) ends (Var 'Y'))
  (OR (not ('B' 2) ends (Var 'Y'))
  (OR (not ('B' 3) ends (Var 'Y'))
  (OR (not ('B' 1) starts (Var 'Y'))
  (OR (not ('B' 2) starts (Var 'Y'))
  (OR (not ('B' 3) starts (Var 'Y'))
)
( 
 (('B' 1) is (('B' 0) (i1 1) (const 0))) 
 (('B' 2) is (('B' 0) (i2 1) (const 1))) 
 (('B' 3) is (('B' 0) (i3 1) (const 0))) 
)
\end{minted}
\item В блок коллапсирует только \verb|Y|:
\begin{minted}[breaklines,breakafter= ]{refal}
(AreEqual 
 (('B' 1) (Var 'X') ('B' 2)) 
 (('B' 3))
)
(
  (OR (not empty (Var 'X')))
  (OR (not ('B' 1) ends (Var 'X'))
  (OR (not ('B' 2) ends (Var 'X'))
  (OR (not ('B' 3) ends (Var 'X'))
  (OR (not ('B' 1) starts (Var 'X'))
  (OR (not ('B' 2) starts (Var 'X'))
  (OR (not ('B' 3) starts (Var 'X'))
)
( 
 (('B' 1) is (('B' 0) (i1 1) (const 0))) 
 (('B' 2) is (('B' 0) (i2 1) (const 0))) 
 (('B' 3) is (('B' 0) (i3 1) (const 1))) 
)
\end{minted}
\item Ничего не коллапсирует:
\begin{minted}[breaklines,breakafter= ]{refal}
(AreEqual 
 (('B' 1) (Var 'X') ('B' 2)) 
 (('B' 3) (Var 'Y') ('B' 4))
)
(
  (OR (not empty (Var 'X')))
  (OR (not empty (Var 'Y')))
  (OR (not ('B' 1) ends (Var 'X'))
  (OR (not ('B' 2) ends (Var 'X'))
  (OR (not ('B' 3) ends (Var 'X'))
  (OR (not ('B' 4) ends (Var 'X'))
  (OR (not ('B' 1) starts (Var 'X'))
  (OR (not ('B' 2) starts (Var 'X'))
  (OR (not ('B' 3) starts (Var 'X'))
  (OR (not ('B' 4) starts (Var 'X'))
  (OR (not ('B' 1) ends (Var 'Y'))
  (OR (not ('B' 2) ends (Var 'Y'))
  (OR (not ('B' 3) ends (Var 'Y'))
  (OR (not ('B' 4) ends (Var 'Y'))
  (OR (not ('B' 1) starts (Var 'Y'))
  (OR (not ('B' 2) starts (Var 'Y'))
  (OR (not ('B' 3) starts (Var 'Y'))
  (OR (not ('B' 4) starts (Var 'Y'))
)
 (('B' 1) is (('B' 0) (i1 1) (const 0))) 
 (('B' 2) is (('B' 0) (i2 1) (const 0))) 
 (('B' 3) is (('B' 0) (i3 1) (const 1))) 
 (('B' 4) is (('B' 0) (i4 1) (const 0))) 
)
\end{minted}

\end{enumerate}
\item Пусть далее рассматривается опять ветка 4: \mintinline{refal}{(Pick 4)}, и аналогичные рассмотренным ранее подстановки:

\mintinline{refal}{(Subst i1 ((i3 1) (const 1)))}

\mintinline{refal}{(Subst i2 ((i4 1) (const 0)))}

После такого преобразования останется только уравнение 

\mintinline{refal}{(AreEqual ((Var 'X')) ((Var 'Y')))} 

с условиями непустоты, и интерактивный режим должен сообщить, что минимальное решение на этой ветке не существует (т.к. остались только переменные).

\item Если бы рассматривалась первая ветка, то следующая подстановка:

\mintinline{refal}{(Subst i1 ((i2 1) (const 1)))}

привела бы к сообщению о нахождении решения.
\end{itemize}


\end{document} 