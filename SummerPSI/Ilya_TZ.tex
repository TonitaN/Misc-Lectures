\documentclass[12pt]{article}
\usepackage{color}
\usepackage[newfloat]{minted}
\usepackage{amsfonts,latexsym,amsthm}
%\usepackage{times}
\usepackage[scaled=.92]{helvet}
\usepackage[all,2cell,knot,poly]{xy}
\usepackage{amssymb,array,amscd,enumerate}
\usepackage[T1,T2A]{fontenc}
\usepackage[utf8]{inputenc}
\usepackage[english,russian]{babel}
%\usepackage[cp1251]{inputenc}
%\usepackage{pscyr}
\def\logand{\mathrel{\&}}
\def\logor{\mathrel{\vee}}
\def\logimpl{\mathrel{\Rightarrow}}
\def\lognot{\mathop{\neg}}
\def\Kcm{\mathbf{K}}
\def\Icm{\mathbf{I}}
\def\Scm{\mathbf{S}}
\def\amodal{\mathop{\Box}}
\def\emodal{\mathop{\Diamond}}
\def\quantall#1{\mathop{\forall #1}}
\def\quantex#1{\mathop{\exists #1}} 
\def\either{\mathrm{either}\,}
\def\Leftt{\mathrm{Left}\;}
\def\Rightt{\mathrm{Right}\;}
\def\snd{\mathrm{snd}\;}
\def\fst{\mathrm{fst}\;}
\def\iff{\mathrel{\Leftrightarrow}}
%
\def\real{\mathop{\rm real}\nolimits}   
\def\Mult{\mathop{\rm Mult}\nolimits}   
\def\Inv{\mathop{\rm Inv}\nolimits}   
\def\round{\mathop{\rm round}\nolimits}
\newcommand{\skipper}{\hspace{-7pt}\textcolor{white}{\displaystyle !_{a_a}^{a^{a^{a^a}}}}}
\begin{document}

\section{Внутренние структуры данных}

Базовые структуры данных алгоритма Ежа: переменная, константа и отрицательное условие.

\begin{itemize}
\item Поскольку множество переменных типа строка не меняется, то логично, что переменная представляется просто своим именем: например, $x$ представляется как \mintinline{refal}{(Var 'X')}.
\item Константы могут являться результатами сжатия блоков, поэтому логично представлять их парами: имя и индекс. Например, \mintinline{refal}{('D' 1)}.
\item Индексы дополнительно имеют расшифровки: мультимножества компонент их длины. Например, если \mintinline{refal}{('D' 1)} --- константа, хранящая блок $D^{i_1 + i_2+2}$, где $D$ --- константа исходного уравнения, то расшифровка будет: 

\mintinline{refal}{('D' 1) is (('D' 0) (i1 1)(i2 1)(const 2))}.
\item Отрицательное условие накладывается на переменную и имеет форму КНФ, где каждый литерал --- это отрицательное утверждение, чем не может кончаться или начинаться переменная, или утверждение о непустоте значения переменной. 

Например, \begin{minted}[breaklines,breakafter= ]{refal}
(OR 
    (not ('D' 1) ends (Var 'X')) 
    (not ('A' 2) starts (Var 'Y'))
)
\end{minted}
\end{itemize}

\section{Базовые операции интерактивного режима}
\begin{itemize}
\item Команда \mintinline{refal}{(PairComp C1 C2)}. Применить сжатие пар: 

\mintinline{refal}{<PairComp (/*Const1*/) (/*Const2*/) (/*EquationData*/)>}. 

Константы не должны совпадать. Результатом сжатия пар станет нумерованный набор новых уравнений с условиями (таким образом, сжатие пар преобразует одно уравнение к множеству уравнений).
\item Команда \mintinline{refal}{(BlockComp C)}. Применить сжатие блоков: 

\mintinline{refal}{<BlockComp (/*Const*/) (/*EquationData*/)>}. 

Аналогично, результат --- нумерованный набор новых уравнений с условиями.
\item Команда \mintinline{refal}{(Pick i)}. Выбрать из нумерованного списка уравнение, соответствующее номеру: \mintinline{refal}{<Pick /*Number*/ /*EquationSet*/>}. 

Действие позволяет перейти от множества уравнений к единственному.
\item Команда \mintinline{refal}{(Subst i1 (/* Multiset */))}. Осуществить подстановку мультимножества компонент, представленного в команде, вместо компоненты \mintinline{refal}{i1}.   
\end{itemize}

\section{Возможные результаты вычисления шага}

\begin{itemize}
\item Если в результате вычислений получилось уравнение вида $\varepsilon=\varepsilon$ или уравнение, которое сводится к нему после вычёркивания всех переменных, тогда объявить, что решение найдено.
\item Если в результате вычислений получилось уравнение, содержащее только переменные, причём хотя бы одна переменная имеет условие непустоты, тогда объявить, что на ветке нет минимального решения (остались только неявные сжатия).
\item Если команда противоречит условиям (например, требуется повторно сжимать блоки, с которых ничего начинаться не может, или требуется сжать блок, содержащий константу, не встречающуюся в уравнении) --- выдать сообщение о некорректном шаге.
\item В противном случае результатом вычислений станет одно или несколько состояний (т.е. уравнений в паре с условиями).
\end{itemize}

\section{Построение новых констант в \texttt{BlockComp} и \texttt{PairComp}}

Вопрос о том, ограничено ли множество используемых в процессе преобразования констант некоторой разумной (полиномиально зависящей от размера уравнения) величиной, пока открыт. Будем считать, что мы такое ограничение не ставим. Поэтому в дальнейшем нет смысла связывать новые имена констант со старыми, а лучше порождать их по очереди.

Чтобы код был написан полностью в функциональном стиле, достаточно хранить в общем состоянии последний используемый индекс блоков и последнюю использованную новую константу. Далее порождать на их основе очередную константу, например, пользуясь алфавитным сдвигом и комбинацией встроенных функций \mintinline{refal}{<Explode s.X>} -- \mintinline{refal}{<Implode e.X>} и \mintinline{refal}{<Numb e.N>} -- \mintinline{refal}{<Symb s.N>}.

\begin{itemize}
\item Функция \mintinline{refal}{<Explode s.X>} разбирает многобуквенную константу на символы и возвращает строку: \mintinline{refal}{Term} -> \mintinline{refal}{'Term'}.
\item Функция \mintinline{refal}{<Implode e.X>} собирает символы в многобуквенную константу (не любая последовательность символов может ей стать, выбирается самая длинная подпоследовательность значения \mintinline{refal}{e.X}): \mintinline{refal}{'Term 123'} -> \mintinline{refal}{Term '123'}.
\item Функция \mintinline{refal}{<Symb s.X>} разбирает число на символы и возвращает строку: \mintinline{refal}{123} -> \mintinline{refal}{'123'}. Функция \mintinline{refal}{<Numb e.X>} действует обратным образом.
\end{itemize}

Простейший код для порождения новых индексов блоков всего лишь увеличивает индекс константы. Константой здесь считаем литерал (многобуквенный терм без кавычек), который начинается на букву, а следом за ней обязательно идёт цифра. Например, \mintinline{refal}{i19} (не путаем со строкой \mintinline{refal}{'i19'}).

\begin{minted}{refal}
GetNewIndex {
 s.OldIndex
 , <Explode s.OldIndex> : s.Letter e.Digits
 , <Symb <Add <Numb e.Digits> 1>> : e.NewNumber
  = <Implode s.Letter e.NewNumber>;
}
\end{minted}

Улучшением этого базового варианта может быть вариант, который меняет сначала букву (выбирая в строке возможных вариантов ближайшую справа от текущей), а если не задействованных букв уже не осталось --- то цифру, возвращая указатель на букву в начало алфавита.

\begin{minted}[breaklines, linenos]{refal}
/* В этом варианте есть тонкость: начальным буквам стоит присвоить индексы с нулём, а порождённым - начиная с единицы, чтобы не путаться. Реализацию этой тонкости оставим за кадром ТЗ. */
GetNewIndexUpdated {
 s.OldIndex
 , <Explode s.OldIndex> : s.Letter e.Digits
 , <CharAplhabet> : {
    e.A1 s.Letter s.NextLetter e.A2 /* Можно выбрать следующую по очереди букву */
        = <Implode s.NextLetter e.Digits>;
    s.FirstLetter e.A1 s.Letter /* Буква уже последняя --- переходим на первую и увеличиваем число */
    , <Symb <Add <Numb e.Digits> 1>> : e.NewNumber
        = <Implode s.FirstLetter e.NewNumber>;
  };
}
\end{minted}

Заметим, что эти же приёмы можно использовать для порождения не только индексов блоков, но и букв --- то есть пар вида \mintinline{refal}{(s.Name s.Number)}. Всё, что нужно будет поменять --- это не делать первый вызов функции \mintinline{refal}{<Explode s.OldIndex>} с последующей сборкой символов в число (и обратное ему преобразование), а просто сдвигать букву \mintinline{refal}{s.Name} и увеличивать числовой индекс \mintinline{refal}{s.Number}.
 
\section{Нулевые значения переменных}
 
В дальнейшем подразумевается следующее допущение. Все извлечённые блоки по умолчанию полагаются непустыми --- пустыми их может сделать только явная подстановка \mintinline{refal}{(SubstIndex ik ((Const 0)))} (или композиция таких подстановок), за которую отвечает пользователь. Более того, все извлечённые блоки по умолчанию предполагаются неравными --- в противном случае от пользователя ожидается подстановка, которая явно утверждает обратное.

Однако в случае вызова \mintinline{refal}{<PairComp /* args */>} может получиться, что потребуется автоматически разобрать и случай пустой подстановки в переменную, чтобы выявить все возможности сжатия перекрёстных или явных вхождений. Например, как в следующем уравнении (назовём его Уравнением 1): 

\begin{minted}{refal}
(AreEqual 
   ((Var 'X') ('B' 0) (Var 'Y') ('A' 0))
   ((Var 'Y') (Var 'Y') ('B' 0) ('A' 0) (Var 'X'))
)
(
   (OR (not empty (Var 'X')))
)
(/* Блок уравнений на компоненты пуст */)
\end{minted}
 
Вызов \mintinline{refal}{<PairComp ('B' 0) ('A' 0) (/* Уравнение 1 */)>} порождает только разбор случаев по переменной \verb|Y|. Полагаем, что новой константой, сжимающей \mintinline{refal}{('B' 0)('A' 0)}, является ('A' 1). 

\begin{enumerate}
\item Случай, когда \verb|Y| кончается на \mintinline{refal}{('B' 0)} и начинается на \mintinline{refal}{('A' 0)}:

\begin{minted}{refal}
(AreEqual 
   ((Var 'X') ('A' 1) (Var 'Y') ('A' 1))
   (('A' 0) (Var 'Y') ('A' 1) (Var 'Y') ('B' 0)('A' 1) (Var 'X'))
)
(
   (OR (not empty (Var 'X')))
)
(/* Блок уравнений на компоненты пуст */)
\end{minted}
 
\item Случай, когда \verb|Y| кончается на \mintinline{refal}{('B' 0)}, но не начинается на \mintinline{refal}{('A' 0)}:

\begin{minted}{refal}
(AreEqual 
   ((Var 'X') ('B' 0) (Var 'Y') ('A' 1))
   ((Var 'Y') ('B' 0) (Var 'Y') ('B' 0)('A' 1) (Var 'X'))
)
(
   (OR (not empty (Var 'X')))
   (OR (not ('A' 0) starts (Var 'Y')))
)
(/* Блок уравнений на компоненты пуст */)
\end{minted}

\item Случай, когда \verb|Y| не кончается на \mintinline{refal}{('B' 0)}, но начинается на \mintinline{refal}{('A' 0)}:

\begin{minted}{refal}
(AreEqual 
   ((Var 'X') ('A' 1) (Var 'Y') ('A' 0))
   (('A' 0) (Var 'Y') ('A' 0) (Var 'Y') ('A' 1) (Var 'X'))
)
(
   (OR (not empty (Var 'X')))
   (OR (not ('B' 0) ends (Var 'Y')))
)
(/* Блок уравнений на компоненты пуст */)
\end{minted}

\item Случай, когда \verb|Y| не кончается на \mintinline{refal}{('B' 0)} и не начинается на \mintinline{refal}{('A' 0)}:

\begin{minted}{refal}
(AreEqual 
   ((Var 'X') ('B' 0) (Var 'Y') ('A' 0))
   ((Var 'Y') (Var 'Y') ('A' 1) (Var 'X'))
)
(
   (OR (not empty (Var 'X')))
   (OR (not ('A' 0) starts (Var 'Y')))
   (OR (not ('B' 0) ends (Var 'Y')))
)
(/* Блок уравнений на компоненты пуст */)
\end{minted}

\item Случай, когда \verb|Y| пуст:

\begin{minted}{refal}
(AreEqual 
   ((Var 'X') ('A' 1))
   (('A' 1) (Var 'X'))
)
(
   (OR (not empty (Var 'X')))
)
(/* Блок уравнений на компоненты пуст */)
\end{minted}
\end{enumerate}

Последний случай принципиально отличается от всех предыдущих и не может быть сведён к ним. Кажется, что он похож на случай 4, но в случае 4 есть очевидное противоречие, если рассмотреть его образ в линейной целочисленной арифметике по модулю $A_1$, поскольку предполагается, что буквы $B_0$ и $A_0$, явно входящие в уравнение, уже точно не принадлежат никакому вхождению $B_0 A_0$. В случае же 5 противоречий нет, и он является единственным, приводящим к решению (после подстановки сжатия блоков $A_1$). 
 
\section{Пример}

Покажем пример работы цепочки операций на следующем примере.

Исходное уравнение: 

\begin{minted}[breaklines,breakafter= ]{refal}
(AreEqual 
 ((Var 'X') ('A' 0) ('A' 0)) 
 (('B' 0) (Var 'Y'))
)
(/* Блок условий пуст */)
(/* Блок уравнений на компоненты пуст */)
\end{minted}

\begin{itemize}

\item Применяем \mintinline{refal}{(BlockComp ('A' 0))}. Получаем набор уравнений:
\begin{enumerate}
\item Случай, когда все переменные коллапсируют в блоки:
\begin{minted}[breaklines,breakafter= ]{refal}
(AreEqual 
 (('A' 1)) 
 (('B' 0) ('A' 2))
)
(/* Блок условий пуст */)
( 
 (('A' 1) is ('A' (i1 1) (const 2))) 
 (('A' 2) is ('A' (i2 1) (const 0))) 
)
\end{minted}
\item В блок коллапсирует только \verb|X|:
\begin{minted}[breaklines,breakafter= ]{refal}
(AreEqual 
 (('A' 1)) 
 (('B' 0) ('A' 2) (Var 'Y') ('A' 3))
)
(
  (OR (not empty (Var 'Y')))
  (OR (not ('A' 1) ends (Var 'Y'))
  (OR (not ('A' 2) ends (Var 'Y'))
  (OR (not ('A' 3) ends (Var 'Y'))
  (OR (not ('A' 1) starts (Var 'Y'))
  (OR (not ('A' 2) starts (Var 'Y'))
  (OR (not ('A' 3) starts (Var 'Y'))
)
( 
 (('A' 1) is (('A' 0) (i1 1) (const 2))) 
 (('A' 2) is (('A' 0) (i2 1) (const 0))) 
 (('A' 3) is (('A' 0) (i3 1) (const 0))) 
)
\end{minted}
\item В блок коллапсирует только \verb|Y|:
\begin{minted}[breaklines,breakafter= ]{refal}
(AreEqual 
 (('A' 1) (Var 'X') ('A' 2)) 
 (('B' 0) ('A' 3))
)
(
  (OR (not empty (Var 'X')))
  (OR (not ('A' 1) ends (Var 'X'))
  (OR (not ('A' 2) ends (Var 'X'))
  (OR (not ('A' 3) ends (Var 'X'))
  (OR (not ('A' 1) starts (Var 'X'))
  (OR (not ('A' 2) starts (Var 'X'))
  (OR (not ('A' 3) starts (Var 'X'))
)
( 
 (('A' 1) is (('A' 0) (i1 1) (const 0))) 
 (('A' 2) is (('A' 0) (i2 1) (const 2))) 
 (('A' 3) is (('A' 0) (i3 1) (const 0))) 
)
\end{minted}
\item Ничего не коллапсирует:
\begin{minted}[breaklines,breakafter= ]{refal}
(AreEqual 
 (('A' 1) (Var 'X') ('A' 2)) 
 (('B' 0) ('A' 3) (Var 'Y') ('A' 4))
)
(
  (OR (not empty (Var 'X')))
  (OR (not empty (Var 'Y')))
  (OR (not ('A' 1) ends (Var 'X'))
  (OR (not ('A' 2) ends (Var 'X'))
  (OR (not ('A' 3) ends (Var 'X'))
  (OR (not ('A' 4) ends (Var 'X'))
  (OR (not ('A' 1) starts (Var 'X'))
  (OR (not ('A' 2) starts (Var 'X'))
  (OR (not ('A' 3) starts (Var 'X'))
  (OR (not ('A' 4) starts (Var 'X'))
  (OR (not ('A' 1) ends (Var 'Y'))
  (OR (not ('A' 2) ends (Var 'Y'))
  (OR (not ('A' 3) ends (Var 'Y'))
  (OR (not ('A' 4) ends (Var 'Y'))
  (OR (not ('A' 1) starts (Var 'Y'))
  (OR (not ('A' 2) starts (Var 'Y'))
  (OR (not ('A' 3) starts (Var 'Y'))
  (OR (not ('A' 4) starts (Var 'Y'))
)
 (('A' 1) is (('A' 0) (i1 1) (const 0))) 
 (('A' 2) is (('A' 0) (i2 1) (const 2))) 
 (('A' 3) is (('A' 0) (i3 1) (const 0))) 
 (('A' 4) is (('A' 0) (i4 1) (const 0))) 
)
\end{minted}

\end{enumerate}
\item Выбираем четвёртый случай: \mintinline{refal}{(Pick 4)}. Осуществляем ряд подстановок индексов: 

\mintinline{refal}{(Subst i1 ((const 0)))}

\mintinline{refal}{(Subst i3 ((const 0)))}

\mintinline{refal}{(Subst i4 ((i2 1) (const 2)))}

Интерактивный режим сам сократит одинаковые суффиксы и удалит ненужные зависимости, ведь констант \mintinline{refal}{('A' k)} после этого в уравнении не останется:

\begin{minted}[breaklines,breakafter= ]{refal}
(AreEqual 
 ((Var 'X')) 
 (('B' 0) (Var 'Y'))
)
(/* Тут ничего нет, кроме условий на непустоту: 
    все упоминания констант, связанных с 'A', 
    исчезли из уравнения, и нет никаких связей этих 
    констант ни с чем, кроме как с собой, в блоке
    индексов, значит, и условия на них уже не нужны */
  (OR (not empty (Var 'X')))
  (OR (not empty (Var 'Y')))    
 )
(/* Тут тоже ничего нет --- всё сократилось */)
\end{minted}
\item Сжимаем блоки \mintinline{refal}{('B' 0)}. Опять получаем четыре случая.
\begin{enumerate}
\item Случай, когда все переменные коллапсируют в блоки:
\begin{minted}[breaklines,breakafter= ]{refal}
(AreEqual 
 (('B' 1)) 
 (('B' 2))
)
(/* Блок условий пуст */)
( 
 (('B' 1) is ('B' (i1 1) (const 0))) 
 (('B' 2) is ('B' (i2 1) (const 1))) 
)
\end{minted}
\item В блок коллапсирует только \verb|X|:
\begin{minted}[breaklines,breakafter= ]{refal}
(AreEqual 
 (('B' 1)) 
 (('B' 2) (Var 'Y') ('B' 3))
)
(
  (OR (not empty (Var 'Y')))
  (OR (not ('B' 1) ends (Var 'Y'))
  (OR (not ('B' 2) ends (Var 'Y'))
  (OR (not ('B' 3) ends (Var 'Y'))
  (OR (not ('B' 1) starts (Var 'Y'))
  (OR (not ('B' 2) starts (Var 'Y'))
  (OR (not ('B' 3) starts (Var 'Y'))
)
( 
 (('B' 1) is (('B' 0) (i1 1) (const 0))) 
 (('B' 2) is (('B' 0) (i2 1) (const 1))) 
 (('B' 3) is (('B' 0) (i3 1) (const 0))) 
)
\end{minted}
\item В блок коллапсирует только \verb|Y|:
\begin{minted}[breaklines,breakafter= ]{refal}
(AreEqual 
 (('B' 1) (Var 'X') ('B' 2)) 
 (('B' 3))
)
(
  (OR (not empty (Var 'X')))
  (OR (not ('B' 1) ends (Var 'X'))
  (OR (not ('B' 2) ends (Var 'X'))
  (OR (not ('B' 3) ends (Var 'X'))
  (OR (not ('B' 1) starts (Var 'X'))
  (OR (not ('B' 2) starts (Var 'X'))
  (OR (not ('B' 3) starts (Var 'X'))
)
( 
 (('B' 1) is (('B' 0) (i1 1) (const 0))) 
 (('B' 2) is (('B' 0) (i2 1) (const 0))) 
 (('B' 3) is (('B' 0) (i3 1) (const 1))) 
)
\end{minted}
\item Ничего не коллапсирует:
\begin{minted}[breaklines,breakafter= ]{refal}
(AreEqual 
 (('B' 1) (Var 'X') ('B' 2)) 
 (('B' 3) (Var 'Y') ('B' 4))
)
(
  (OR (not empty (Var 'X')))
  (OR (not empty (Var 'Y')))
  (OR (not ('B' 1) ends (Var 'X'))
  (OR (not ('B' 2) ends (Var 'X'))
  (OR (not ('B' 3) ends (Var 'X'))
  (OR (not ('B' 4) ends (Var 'X'))
  (OR (not ('B' 1) starts (Var 'X'))
  (OR (not ('B' 2) starts (Var 'X'))
  (OR (not ('B' 3) starts (Var 'X'))
  (OR (not ('B' 4) starts (Var 'X'))
  (OR (not ('B' 1) ends (Var 'Y'))
  (OR (not ('B' 2) ends (Var 'Y'))
  (OR (not ('B' 3) ends (Var 'Y'))
  (OR (not ('B' 4) ends (Var 'Y'))
  (OR (not ('B' 1) starts (Var 'Y'))
  (OR (not ('B' 2) starts (Var 'Y'))
  (OR (not ('B' 3) starts (Var 'Y'))
  (OR (not ('B' 4) starts (Var 'Y'))
)
 (('B' 1) is (('B' 0) (i1 1) (const 0))) 
 (('B' 2) is (('B' 0) (i2 1) (const 0))) 
 (('B' 3) is (('B' 0) (i3 1) (const 1))) 
 (('B' 4) is (('B' 0) (i4 1) (const 0))) 
)
\end{minted}

\end{enumerate}
\item Пусть далее рассматривается опять ветка 4: \mintinline{refal}{(Pick 4)}, и аналогичные рассмотренным ранее подстановки:

\mintinline{refal}{(Subst i1 ((i3 1) (const 1)))}

\mintinline{refal}{(Subst i2 ((i4 1) (const 0)))}

После такого преобразования останется только уравнение 

\mintinline{refal}{(AreEqual ((Var 'X')) ((Var 'Y')))} 

с условиями непустоты, и интерактивный режим должен сообщить, что минимальное решение на этой ветке не существует (т.к. остались только переменные).

\item Если бы рассматривалась первая ветка, то следующая подстановка:

\mintinline{refal}{(Subst i1 ((i2 1) (const 1)))}

привела бы к сообщению о нахождении решения.
\end{itemize}


\end{document} 